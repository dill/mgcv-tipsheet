% !TEX root = mgcv.tex

\begin{block}{Extras}
  \small\renewcommand{\arraystretch}{1.5}
  \begin{tabular}{c >{} p{0.45\linewidth}}
    \inline{gam.mh} & Metropolis-Hastings sampling of the posterior\\
    \inline{concurvity} & Assess concurvity between terms\\
    \inline{gam.vcomp} & Random effects style output\\
    \inline{gamSim} & Simulate GAM-type data\\
    \inline{inSide}/\inline{in.out} & point-in-polygon test\\
    \inline{jagam} & Generate JAGS/Nimble code\\
    \inline{new.name} & Generate a variable name\\
    \inline{place.knots} & Place knots evenly\\
    \inline{rmvn} & Generate multivariate normal deviates\\
  \end{tabular}
\end{block}

\begin{block}{Extra help}
  \small\renewcommand{\arraystretch}{1.5}
  \begin{tabular}{c >{} p{0.45\linewidth}}
    \inline{?gam.models} & Fitting fancy models\\
    \inline{?linear.functionals} & \\
    \inline{?random.effects} & \\
    \inline{?mgcv.FAQ} & frequently asked questions\\
    \inline{?mgcv.parallel} & Info on parallelisation\\
    \inline{?missing.data} & \\
    \inline{?choose.k} & How to select basis size\\
    \inline{?one.se.rule} & \\
  \end{tabular}
\end{block}


\begin{block}{Other packages}
  \small\renewcommand{\arraystretch}{1.5}
  \begin{tabular}{c >{} p{0.45\linewidth}}
    \inline{scam} & \\
    \inline{gratia} & \\
    \inline{mgcViz} & \\
    \inline{qgam} & \\
    \inline{gamm4} & \\
  \end{tabular}
\end{block}

% !TEX root = mgcv.tex


\begin{block}{Knots and basis complexity}
  General strategy: check \texttt{k} and double if too small.\\
  When do we know \texttt{k} is too small?

        \begin{exampleblock}{Example}
          \begin{Rbg}
> gam.check(b)

Method: GCV   Optimizer: magic
Smoothing parameter selection converged
after 12 iterations.
The RMS GCV score gradient at convergence
was 1.739918e-07 .
The Hessian was positive definite.
Model rank =  37 / 37 

Basis dimension (k) checking results. 
Low p-value (k-index<1) may
indicate that k is too low, especially if
edf is close to k'.

       k' edf k-index p-value
s(x0) 9.0 2.5    1.04    0.85
s(x1) 9.0 2.4    1.03    0.69
s(x2) 9.0 7.7    0.97    0.28
s(x3) 9.0 1.0    1.03    0.68

          \end{Rbg}
        \end{exampleblock}
Just as it says, check the \texttt{p-value} and \texttt{k-index} columns! Double\texttt{k} if necessary.

\end{block}


\begin{block}{\texttt{predict}}
  \texttt{type=} argument changes the type of prediction
  \renewcommand{\arraystretch}{1.411}\hspace{-17.5pt}
  \begin{tabular}{>{\centering}m{0.26\linewidth} >{\centering\arraybackslash}m{0.70\linewidth}}
  default & on the link scale\\
  \inlc{"response"} & to put on the response scale\\
  \inlc{"iterms"} & to give per term predictions\\
  \inlc{"lpmatrix"} & for a prediction matrix\\
  \end{tabular}
  \vskip-1ex
\end{block}




\begin{block}{Fitting criterion \texttt{method=}}
  \renewcommand{\arraystretch}{1.411}\hspace{-17.5pt}
  \begin{tabular}{>{\centering}m{0.26\linewidth} >{\centering\arraybackslash}m{0.70\linewidth}}
    \inlc{"GCV.Cp"} & Generalized cross validation, default\\
    \inlc{"REML"} & REstricted Maximum Likelihood, preferred\\
    \inlc{"ML"} & Maximum Likelihood\\
    \inlc{"NCV"} & Neighbourhood Cross-Validation\\
  \end{tabular}
  \vskip-1ex
\end{block}

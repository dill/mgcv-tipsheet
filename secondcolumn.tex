% !TEX root = mgcv.tex

\begin{block}{Basic operation of \texttt{gam}}
  \begin{subblock}{\texttt{formula=}}
    We can write a model formula in \mgcv just as we can when we use \texttt{lm} or \texttt{glm}, with some additions.\br
    \inline{s()} is the general setup for a smooth.\br
    \inline{te()} allows us to construct an interaction using a tensor product.\br
    \vskip-1ex
  \end{subblock}
  
  \begin{subblock}{Response distribution \texttt{family=}}
    \begin{tabular}{>{\centering}m{0.40\linewidth} >{\centering\arraybackslash}m{0.6\linewidth}}
    Binomial         &  \texttt{binomial}\\
    Normal         &  \texttt{gaussian}\\
    Gamma            &  \texttt{Gamma}\\
    Inverse normal &  \texttt{inverse.gaussian}\\
    Poisson          &  \texttt{poisson}\\
    Quasi            &  \texttt{quasi}\\
    Quasi-binomial    &  \texttt{quasibinomial}\\
    Quasi-Poisson     &  \texttt{quasipoisson}\\
    Tweedie       &  \texttt{tw}/\texttt{Tweedie}\\
    Negative binomial        &  \texttt{nb}/\texttt{negbin}\\
    Beta            &  \texttt{betar}\\
    Censored normal            &  \texttt{cnorm}\\
    Ordered categorical             &  \texttt{ocat}\\
    Scaled $t$             &  \texttt{scat}\\
    Zero inflated Poisson              &  \texttt{ziP}\\
    Zero inflated Poisson location-scale   &  \texttt{ziplss}\\
    Cox proportional hazards & \texttt{cox.ph}\\
    Generalized extreme value location-scale & \inline{gevlss}\\
    Normal location-scale model & \inline{gaulss}\\
    Multivariate normal & \inline{mvn}\\
    Gamma location-scale & \inline{gammals}\\
    Gumbel location-scale & \inline{gumbls}\\
    Multinomial & \inline{multinom}\\
    Tweedie location-scale & \inline{twlss}\\
    General family               &  \texttt{gfam}
    \end{tabular}
    \vskip-1ex
  \end{subblock}
\end{block}
